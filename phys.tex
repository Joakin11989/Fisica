\usepackage[spanish]{babel}
\usepackage[utf8]{inputenc}
\usepackage[a4paper,top=2.5cm,bottom=2.5cm,left=1.7cm,right=1.7cm]{geometry}
\usepackage{mdframed}
\usepackage{xcolor}
\usepackage{amsmath, amsfonts, amssymb}
\usepackage{tikz}
\usepackage{graphicx}
\graphicspath{{images/}}
\usepackage{cancel}
\usepackage{multirow}
\usepackage{hyperref}
\hypersetup{
    colorlinks=true,        % Colorea los enlaces en lugar de usar cajas
    linkcolor=blue,         % Color de los enlaces internos (por ejemplo, índice)
    urlcolor=cyan           % Color de los enlaces externos (por ejemplo, URLs)
}
\usepackage{appendix}
\definecolor{gf}{RGB}{220,220,220}

\title{\textbf{F\'isica}}
\author{Joaqu\'in Parra S. \\ FCFM UdeC}

\begin{document}
\maketitle
\cleardoublepage
\frontmatter
\chapter*{Prólogo}  % O Prefacio
 Este apunte nace con el fin de proporcionar una herramienta externa de apoyo a los estudiantes de ense\~{n}anza media del Colegio 
 Seminario Padre Alberto Hurtado de Chill\'an (del cual alguna vez fui estudiante) para sus cursos de f\'isica, y tambi\'en para posibles
 competencias (tanto internas como externas) en las que quisieran participar, aunque la distribuci\'on de este mismo se deja a manos del lector.
 Adem\'as de ser un texto de apoyo para estudiantes, el presente documento busca recuperar el espíritu pre-pandemia de los electivos de f\'isica 
 y matem\'aticas. Aqu\'i se proporcionan herramientas que ser\'an de utilidad para estudiantes que quieran seguir un camino en las \'areas de 
 ciencias f\'isicas y matem\'aticas, sin dejar de lado a la gente que recurra a \'el para salvar una nota.
 Con el paso del tiempo, se ir\'an agregando contenidos y se ir\'a perfeccionando este documento, el cual se escribi\'o utilizando el sistema
 {\LaTeX}, el cual se recomienda encarecidamente aprender si planea caminar por la senda de las ciencias puras.
 \cleardoublepage
\tableofcontents
\mainmatter

\chapter{Mediciones}
 Previo al estudio de \emph{la f\'isica} como tal, es necesario hablar de como \emph{se mide} en f\'isica. Al final de cuentas, todas las 
 ecuaciones f\'isicas son aplicadas sobre par\'ametros los cuales han de ser medidos previamente. 
 Partiremos definiendo a la \textbf{\emph{medici\'on}} como \textbf{el acto de calcular el valor n\'umerico de cierta magnitud f\'isica.}\\
 Considere la siguiente situación: 
 \begin{mdframed}[backgroundcolor=gf]
  Usted tiene un CD de un radio desconocido, el cual se propone a calcular. Para esto, usted escoge una regla
  y procede a medir la distancia desde el centro de dicho CD hasta un extremo. Ahora, considere que dicha regla arrojó que el radio del CD era
  de unos ``6 $cm$". ¿Podría afirmar que ese es el valor exacto del radio del CD? Muy probablemente su regla esté graduada hasta los milímetros.
  Considerando la misma regla, ¿Podría afirmar que el radio del CD no es de 6,00000001 $cm$?
 \end{mdframed}
 La finalidad de este capítulo es entregar las bases para que usted pueda operar magnitudes de forma correcta y ordenada. Si bien durante su 
 paso por la enseñanza media este cap\'itulo no ser\'a de mayor relevancia\footnote{Si planea participar en las Olimpiadas de F\'isica, le 
 recomiendo que \textbf{si} considere estudiar este cap\'itulo.},  durante su vida universitaria como estudiante de STEM (Si eso es lo 
 que quiere) tendr\'a que acostumbrarse a su uso.
 
 \section{Cifras significativas}
 
 
 
 \section{Unidades de medida y S.I.}
\chapter{Cinemática}
 \section{Movimiento en una dimensi\'on}
  \subsection{MRU}
  \subsection{MRUA}
 \section{Movimiento en 2 dimensiones}
  \subsection{Movimiento parab\'olico}
  \subsection{Movimiento circunferencial uniforme}
  \subsection{MCUA}
  \subsection{Movimiento general en 2 dimensiones}
  \section{Movimiento general en 3 dimensiones}
\chapter{Dinámica}
 \section{Leyes de Newton}
  \subsection{Ley de inercia}
  \subsection{Ley fundamental de la din\'amica}
  \subsection{Ley de acci\'on-reacci\'on}
  \section{Fuerzas t\'ipicas y aplicaciones}
  \section{Din\'amica rotacional}
\chapter{Cantidad de Movimiento, Momento Angular y S\'olido Rígido}
aaaaa
\chapter{Energ\'ia}
aaaaa
\chapter{Fluidos}
 \section{Hidr\'ostatica}
  \subsection{Presi\'on}
  \subsection{Empuje y equilibrio hidr\'ostatico}
 \section{Hidrodin\'amica}
\chapter{Oscilaciones y Ondas}
aaaaaa
\chapter{Calorimetr\'ia y Termodinámica}
aaaa
\section{Electricidad}
aaaaa
\chapter{Magnetismo}
aaaaa
\chapter{F\'isica Moderna}
 \section{Conceptos de Relatividad Especial}
 \section{Conceptos de Relatividad General} 
 \section{Conceptos de Mecánica Cuántica}
\chapter{Ap\'endices}
\begin{appendices}
\chapter{C\'alculo Diferencial e Integral}
 \section{\'Algebra y Trigonometr\'ia}
 \section{L\'imites}
 \section{Derivadas}
 \section{Integrales}
\chapter{\'Algebra de Vectores, Matrices e Introducci\'on al C\'alculo Vectorial}
\end{appendices}
\end{document}
