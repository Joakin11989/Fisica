\documentclass[12pt]{article}
\usepackage[spanish]{babel}
\usepackage[utf8]{inputenc}
\usepackage{amsmath, amsfonts, amssymb}
\usepackage{tikz}
\usepackage{graphicx}
\graphicspath{{images/}}
\usepackage{cancel}
\usepackage{multirow}
\usepackage{hyperref}
\hypersetup{
    colorlinks=true,        % Colorea los enlaces en lugar de usar cajas
    linkcolor=red,         % Color de los enlaces internos (por ejemplo, índice)
    urlcolor=cyan           % Color de los enlaces externos (por ejemplo, URLs)
}
\usepackage{appendix}

\title{\textbf{F\'isica}}
\author{Joaqu\'in Parra S. \\ FCFM UdeC}

\begin{document}
\begin{center}
 {\Huge \textbf{F\'isica\\}}%No abreviar, no subrayar, no usar comillas. Se escribe completamente en mayúsculas
		        \vspace*{1.5\baselineskip}

		\large{\textbf{Por: Joaqu\'in Parra S.}}\\ %No abreviar, no subrayar, no usar comillas. Solo la priemra letra usa mayúsculas
		
        \vspace*{1,5\baselineskip}

		\large{\textbf{FCFM UdeC}}\\ %Nombre como aparece en registro académico
		
		\vspace{1,5\baselineskip}
		
\end{center}
\large{
\begin{abstract}
 Este apunte nace con el fin de proporcionar una herramienta externa de apoyo a los estudiantes de ense\~{n}anza media del Colegio 
 Seminario Padre Alberto Hurtado de Chill\'an (del cual alguna vez fui estudiante) para sus cursos de f\'isica, y tambi\'en para posibles
 competencias (tanto internas como externas) en las que quisieran participar, aunque la distribuci\'on de este mismo se deja a manos del lector.
 Adem\'as de ser un texto de apoyo para estudiantes, el presente documento busca recuperar el espíritu pre-pandemia de los electivos de f\'isica 
 y matem\'aticas. Aqu\'i se proporcionan herramientas que ser\'an de utilidad para estudiantes que quieran seguir un camino en las \'areas de 
 ciencias f\'isicas y matem\'aticas, sin dejar de lado a la gente que recurra a \'el para salvar una nota.
 Con el paso del tiempo, se ir\'an agregando contenidos y se ir\'a perfeccionando este documento, el cual se escribi\'o utilizando el sistema
 {\LaTeX}, el cual se recomienda encarecidamente aprender si planea caminar por la senda de las ciencias puras.
\end{abstract}
}
\newpage
\tableofcontents
\section{Mediciones}
aaaaa

\section{Cinemática}
 \subsection{Movimiento en una dimensi\'on}
  \subsubsection{MRU}
  \subsubsection{MRUA}
\section{Dinámica}
aaaa
\section{Cantidad de Movimiento y Momento Angular}
aaaaa
\section{Energ\'ia}
aaaaa
\section{Oscilaciones y Ondas}
aaaaaa
\section{Calorimetr\'ia y Termodinámica}
aaaa
\section{Electricidad}
aaaaa
\section{Magnetismo}
aaaaa
\section{F\'isica Moderna}
 \subsection{Conceptos de Relatividad Especial}
 \subsection{Conceptos de Relatividad General} 
 \subsection{Conceptos de Mecánica Cuántica}

\begin{appendices}
\section{C\'alculo Diferencial e Integral}
 \subsection{\'Algebra y Trigonometr\'ia}
 \subsection{L\'imites}
 \subsection{Derivadas}
 \subsection{Integrales}
\section{\'Algebra de Vectores, Matrices y Calculo Vectorial}
\end{appendices}
\end{document}